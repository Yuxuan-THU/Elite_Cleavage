% Options for packages loaded elsewhere
\PassOptionsToPackage{unicode}{hyperref}
\PassOptionsToPackage{hyphens}{url}
%
\documentclass[
]{book}
\usepackage{amsmath,amssymb}
\usepackage{iftex}
\ifPDFTeX
  \usepackage[T1]{fontenc}
  \usepackage[utf8]{inputenc}
  \usepackage{textcomp} % provide euro and other symbols
\else % if luatex or xetex
  \usepackage{unicode-math} % this also loads fontspec
  \defaultfontfeatures{Scale=MatchLowercase}
  \defaultfontfeatures[\rmfamily]{Ligatures=TeX,Scale=1}
\fi
\usepackage{lmodern}
\ifPDFTeX\else
  % xetex/luatex font selection
\fi
% Use upquote if available, for straight quotes in verbatim environments
\IfFileExists{upquote.sty}{\usepackage{upquote}}{}
\IfFileExists{microtype.sty}{% use microtype if available
  \usepackage[]{microtype}
  \UseMicrotypeSet[protrusion]{basicmath} % disable protrusion for tt fonts
}{}
\makeatletter
\@ifundefined{KOMAClassName}{% if non-KOMA class
  \IfFileExists{parskip.sty}{%
    \usepackage{parskip}
  }{% else
    \setlength{\parindent}{0pt}
    \setlength{\parskip}{6pt plus 2pt minus 1pt}}
}{% if KOMA class
  \KOMAoptions{parskip=half}}
\makeatother
\usepackage{xcolor}
\usepackage{longtable,booktabs,array}
\usepackage{calc} % for calculating minipage widths
% Correct order of tables after \paragraph or \subparagraph
\usepackage{etoolbox}
\makeatletter
\patchcmd\longtable{\par}{\if@noskipsec\mbox{}\fi\par}{}{}
\makeatother
% Allow footnotes in longtable head/foot
\IfFileExists{footnotehyper.sty}{\usepackage{footnotehyper}}{\usepackage{footnote}}
\makesavenoteenv{longtable}
\usepackage{graphicx}
\makeatletter
\def\maxwidth{\ifdim\Gin@nat@width>\linewidth\linewidth\else\Gin@nat@width\fi}
\def\maxheight{\ifdim\Gin@nat@height>\textheight\textheight\else\Gin@nat@height\fi}
\makeatother
% Scale images if necessary, so that they will not overflow the page
% margins by default, and it is still possible to overwrite the defaults
% using explicit options in \includegraphics[width, height, ...]{}
\setkeys{Gin}{width=\maxwidth,height=\maxheight,keepaspectratio}
% Set default figure placement to htbp
\makeatletter
\def\fps@figure{htbp}
\makeatother
\setlength{\emergencystretch}{3em} % prevent overfull lines
\providecommand{\tightlist}{%
  \setlength{\itemsep}{0pt}\setlength{\parskip}{0pt}}
\setcounter{secnumdepth}{5}
\usepackage{booktabs}
\usepackage{amsthm}
\makeatletter
\def\thm@space@setup{%
  \thm@preskip=8pt plus 2pt minus 4pt
  \thm@postskip=\thm@preskip
}
\makeatother
\ifLuaTeX
  \usepackage{selnolig}  % disable illegal ligatures
\fi
\usepackage[]{natbib}
\bibliographystyle{apalike}
\IfFileExists{bookmark.sty}{\usepackage{bookmark}}{\usepackage{hyperref}}
\IfFileExists{xurl.sty}{\usepackage{xurl}}{} % add URL line breaks if available
\urlstyle{same}
\hypersetup{
  pdftitle={The Historical Context of Confucianism and Neo-Confucianism},
  pdfauthor={Yuxuan Su},
  hidelinks,
  pdfcreator={LaTeX via pandoc}}

\title{The Historical Context of Confucianism and Neo-Confucianism}
\author{Yuxuan Su}
\date{2024-07-05}

\begin{document}
\maketitle

{
\setcounter{tocdepth}{1}
\tableofcontents
}
Brief Research Conclusion

\hypertarget{ux6458ux8981}{%
\chapter{摘要}\label{ux6458ux8981}}

威权政体中资本主义发展的一大挑战是掠夺行为。本文引入新框架,精英分化。

被边缘化的地方干部选择与草根结盟,保护地方经济利益以增加自己的政治生存可能性。

用DID方法分析中国两个省份,发现文革→分权→被边缘化的精英保护企业家对抗中央的极端政策→促进资本主义发展。

更进一步,1940年代以来的精英分化由长远的政治经济影响。

\hypertarget{ux5f15ux8a00}{%
\chapter{引言}\label{ux5f15ux8a00}}

\hypertarget{ux95eeux9898}{%
\section{问题}\label{ux95eeux9898}}

什么情况下,威权精英将避免掠夺,从而培育资本主义?

现有文献提出两个机制:Olson的流寇坐寇理论。目前,政治学者强调正式制度的作用,比如法律和选举,束缚住了独裁者以免掠夺行为。

但是威权制度实现精英间权力共享,对企业家做出可信承诺,这依然是值得怀疑的。

制衡理论高估了商人之于政治家的博弈能力,社会力量太弱的地方不适用于这种理论。

\hypertarget{ux53d1ux73b0}{%
\section{发现}\label{ux53d1ux73b0}}

本文逻辑:精英分化(一些地方官员被边缘化)→被边缘化的精英为了政治生存,选择保护地方经济利益→长期经济增长成为可能

测量:文革时期,浙江省和江苏省资本主义的空间变化。

\textbf{Why Comparable? Natural Experiment:} 它俩很像,都毗邻上海,收入水平高,资本主义发展水平类似,它俩又不太像,一个省的优势群体是另一个省的边缘群体。

边缘精英更有动力与草根结盟,导致经济发展,路径依赖导致改开后的经济腾飞。主导政治派系统治的县城中,地方精英的前途更容易受到上层的摆布,因而他们更倾向于文革到来时支持掠夺性政策。

\hypertarget{ux8d21ux732e}{%
\section{贡献}\label{ux8d21ux732e}}

解释了社会主义国家的资本主义兴起,即一部分政治精英与社会力量抑制了威权的掠夺冲动。

丰富了历史制度影响经济发展的一支文献。

强调地方精英的``权力状态''对精英行为的影响,尤其是在学界广泛关注精英的社会嵌入性的大背景下。

\hypertarget{ux8ba9ux5730ux65b9ux8054ux76dfux8fd0ux8f6cux8d77ux6765ux7cbeux82f1ux5206ux5316ux4e0eux653fux6cbbux751fux5b58}{%
\chapter{让地方联盟运转起来:精英分化与政治生存}\label{ux8ba9ux5730ux65b9ux8054ux76dfux8fd0ux8f6cux8d77ux6765ux7cbeux82f1ux5206ux5316ux4e0eux653fux6cbbux751fux5b58}}

\hypertarget{ux8fb9ux7f18ux5316ux7cbeux82f1}{%
\section{边缘化精英}\label{ux8fb9ux7f18ux5316ux7cbeux82f1}}

边缘化的精英:与高层连接松散,难以向高层表达忠心,地位岌岌可危,容易受到政治攻击。

他们的选择:为了弥补,他们要以草根结盟。

草根的定义:县区级以及的政治家和商人。

边缘化精英的特点:无法持续向从属者输送利益,但可以通过提供对经济利益的庇护的渠道,与草根结盟。

作为交换,草根需要在政治调查来临时,拒绝与边缘化精英的敌对派系合作;而且,边缘精英也有更强的基层动员能力,让他们成为特定社会问题的解决者,增强议价能力。

\hypertarget{ux6838ux5fc3ux7cbeux82f1}{%
\section{核心精英}\label{ux6838ux5fc3ux7cbeux82f1}}

政治生命依赖上级提携,因此需要严格执行哪怕是偏激的政策来表忠。

他们建立了``分配联盟'',可以向政治联盟和关键官员分配利益,而资本主义的兴起对他们的利益产生的打击。

\hypertarget{ux4f5cux4e3aux51b2ux51fbux7684ux6587ux9769}{%
\section{作为冲击的文革}\label{ux4f5cux4e3aux51b2ux51fbux7684ux6587ux9769}}

中国经济腾飞的关键在于边缘精英与草根结盟,核心精英往往不可能与草根结盟,因为这与上级任务抵触。

政治环境对边缘精英和核心精英的选择也会产生影响。如果中央保持一致,草根结盟就会暂时休眠,但当中央出现严重分歧,就是草根结盟的时机。

文革是一个冲击,允许地方精英组织支持者来投入权力斗争,提高了争取草根群体的紧迫性。

\hypertarget{ux8fb9ux7f18ux7cbeux82f1ux7684ux5d1bux8d77ux9769ux547dux53caux5176ux9057ux4ea7}{%
\chapter{边缘精英的崛起:革命及其遗产}\label{ux8fb9ux7f18ux7cbeux82f1ux7684ux5d1bux8d77ux9769ux547dux53caux5176ux9057ux4ea7}}

1949年以前的革命让两省形成了迥然不同的政治格局。

\hypertarget{ux6c5fux82cfux7684ux653fux6cbbux683cux5c40}{%
\section{江苏的政治格局}\label{ux6c5fux82cfux7684ux653fux6cbbux683cux5c40}}

抗日战争时期的华中革命根据地,之后形成了第三野战军。

江苏省的党委长期被来自华中革命根据地的军事精英把持,长期超过70\%的席位。

非华中派构成了江苏政治格局中的少数,也即边缘精英。

华中干部控制了原先为根据地的地区,非华中干部试图在其他地区施加更多影响力。

\hypertarget{ux6d59ux6c5fux7684ux653fux6cbbux683cux5c40}{%
\section{浙江的政治格局}\label{ux6d59ux6c5fux7684ux653fux6cbbux683cux5c40}}

抗日战争期间,浙江的游击队往往各自为战,没有被纳入一个互帮互助的网络中。

1949年,当地游击队解放了三分之一的区域,另外的区域被外来的野战军解放了。

浙江的政治格局长期被两派把持:游击队派干部v.s.野战军派干部与南下干部。

由于与党中央一致,政治天平向野战军派干部倾斜,他们占据了80\%的省级党委席位。

然而游击队干部控制着原先游击队作战的县城。

\hypertarget{ux5206ux91ce}{%
\section{分野}\label{ux5206ux91ce}}

分为``核心派系控制县''vs``边缘派系控制县''

核心派系控制县;江苏的华中派,浙江的南下干部。

边缘派系控制县``江苏的非华中派,浙江的游击队干部。

没有草根的支持,即使是核心派系,也很难推进自己的政策。四清运动时,南下干部要收集游击队派干部的罪证,但游击队干部在当地树大根深,让群众不要配合南下干部,最终南下干部没有得逞。

\hypertarget{ux51b2ux51fbux6587ux9769}{%
\section{冲击:文革}\label{ux51b2ux51fbux6587ux9769}}

文革之后,情况发生了变化,中央要打到一切,即使是核心派系,他们的权力也受到了严峻的挑战,丧失了之前的控制力度。

为了避免``经济主义''的指控,两省党委坚决执行激进政策,核心派系的干部也要有样学样,在自己领导的县城贯彻这样的政策。

然而边缘派系获得了动员群众的正当性,这是1949年以来没有的,这给予了他们反抗核心派系的机会。

为了更好地动员群众,边缘派系必须响应草根的利益诉求,那就是对商品经济活动睁只眼闭只眼。

\hypertarget{ux5b9eux8bc1ux7b56ux7565}{%
\chapter{实证策略}\label{ux5b9eux8bc1ux7b56ux7565}}

\hypertarget{ux68c0ux9a8c}{%
\section{检验}\label{ux68c0ux9a8c}}

希望检验这两个省,核心派系控制县资本主义萎缩,边缘派系控制县资本主义繁荣。

更具体来说,就是边缘派系控制县工业企业的所有权发生根本性转变,而核心派系控制县却没有。

\hypertarget{ux6d4bux91cf}{%
\section{测量}\label{ux6d4bux91cf}}

衡量市场经济活动:私营经济产出份额和私营工业人均收入水平。

识别派系:浙江省,1949之前有游击队活动,就是1,没有就是0。

\hypertarget{ux65b9ux6cd5}{%
\section{方法}\label{ux65b9ux6cd5}}

广义双重差分

私营经济份额 \textasciitilde{} β 边缘派系的虚拟变量 + 地理控制变量 + 县级固定效应

\hypertarget{ux5b9eux8bc1ux7ed3ux679c}{%
\chapter{实证结果}\label{ux5b9eux8bc1ux7ed3ux679c}}

\hypertarget{ux6539ux9769ux65f6ux671fux975eux56fdux6709ux5de5ux4e1aux7684ux5d1bux8d77}{%
\section{改革时期非国有工业的崛起}\label{ux6539ux9769ux65f6ux671fux975eux56fdux6709ux5de5ux4e1aux7684ux5d1bux8d77}}

江苏边缘精英统治的县更有可能发展出充满活力的非国有工业部门。

浙江省边缘派县和核心派县的人均非国有工业产出水平在改革期间出现了分化,并在1978年后继续扩大。

\hypertarget{ux975eux56fdux6709ux90e8ux95e8ux7684ux5174ux8d77ux548cux957fux671fux7ecfux6d4eux7ee9ux6548}{%
\section{非国有部门的兴起和长期经济绩效}\label{ux975eux56fdux6709ux90e8ux95e8ux7684ux5174ux8d77ux548cux957fux671fux7ecfux6d4eux7ee9ux6548}}

改革时期浙江省边缘派县的平均每年增长速度快于核心派县1.75\%,江苏省的核心派县比边缘派县增长慢4.28\%。

浙江省和江苏省的边缘派县的经济发展水平都高于核心派县。

边缘派县的政府公共产品支出明显高于核心派县。

  \bibliography{book.bib,packages.bib}

\end{document}
